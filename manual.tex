\documentclass[dvipdfmx,uplatex,b5j,8pt,nomag*]{jsarticle}
\usepackage{nylecnote}
\usepackage{lipsum, multicol,jlisting,bxjalipsum}
\title{{\tt nylecnote}}
\subtitle{マニュアル}
\author{nozomu-y}

\parindent = 0pt
\def\nycolors{nyRed,nyPink,nyPurple,nyBlue,nyIndigo,nyLightBlue,nyCyan,nyTeal,nyGreen,nyLightGreen,nyLime,nyYellow,nyAmber,nyOrange,nyDeepOrange}
\def\bootstrapcol{primary,secondary,info,success,warning,danger,light,dark}

\titleformat{name=\section,numberless}
{}{}{0pt}
{\secTitle{#1}{}}
\titleformat{name=\subsection,numberless}
{}{}{0pt}
{\subsecTitle{#1}{}}

\begin{document}
\maketitle
\tableofcontents
\newpage

\section{プリアンブル}
\begin{lstlisting}
\documentclass[dvipdfmx,uplatex,nomag*]{jsarticle}
\usepackage[<options>]{nylecnote}
\end{lstlisting}

パッケージのオプションには,{\tt onecolumn, twocolumn, noheader, part}を指定できます.
\begin{description}
    \item[{\tt onecolumn}] 何も指定しない場合のデフォルトです.
        {\tt twocolumn}との両立はできません.
    \item[{\tt twocolumn}] {\tt documentclass}のオプションに{\tt twocolumn}を指定する代わりに{\tt nylecnote}のオプションにして下さい.
        {\tt onecolumn}との両立はできません.
    \item[{\tt noheader}] 各ページのヘッダーを表示しないようにします.
    \item[{\tt part}] ヘッダーにパート名を表示します.
\end{description}

\section{見出し}

\begin{multicols}{2}
\section*{Section}
\subsection*{Subsection}
\subsubsection*{Subsubsection}
\paragraph*{Paragraph}\mbox{}\\

\columnbreak

\begin{lstlisting}
\section*{Section}
\subsection*{Subsection}
\subsubsection*{Subsubsection}
\paragraph*{Paragraph}
\end{lstlisting}

\end{multicols}



\newpage
\section{証明・定理環境}

\begin{multicols}{2}
\begin{proof}
    Write your proof here. 
\end{proof}

\columnbreak

\begin{lstlisting}
\begin{proof}
    Write your proof here. 
\end{proof}
\end{lstlisting}
\end{multicols}

\begin{multicols}{2}
\begin{solution}
    Write your solution here. 
\end{solution}

\columnbreak

\begin{lstlisting}
\begin{solution}
    Write your solution here. 
\end{solution}
\end{lstlisting}
\end{multicols}

\begin{multicols}{2}
\begin{theorem}[Name of theorem (optional)]
    Your theorem here.
\end{theorem}

\columnbreak

\begin{lstlisting}
\begin{theorem}[Name of theorem (optional)]
    Your theorem here.
\end{theorem}
\end{lstlisting}
    
\end{multicols}


\begin{multicols}{2}
\begin{definition}[Name of definition (optional)]
    Your definition here.
\end{definition}

\columnbreak

\begin{lstlisting}
\begin{definition}[Name of definition (optional)]
    Your definition here.
\end{definition}
\end{lstlisting}

\end{multicols}

\begin{multicols}{2}
\begin{lemma}[Name of lemma (optional)]
    Your lemma here.
\end{lemma}

\columnbreak

\begin{lstlisting}
\begin{lemma}[Name of lemma (optional)]
    Your lemma here.
\end{lemma}
\end{lstlisting}
    
\end{multicols}


\begin{multicols}{2}
\begin{hypo}[Name of hypo (optional)]
    Your hypo here.
\end{hypo}

\columnbreak

\begin{lstlisting}
\begin{hypo}[Name of hypo (optional)]
    Your hypo here.
\end{hypo}
\end{lstlisting}
    
\end{multicols}


\begin{multicols}{2}
\begin{example}[Name of example (optional)]
    Your example here.
\end{example}

\columnbreak

\begin{lstlisting}
\begin{example}[Name of example (optional)]
    Your example here.
\end{example}
\end{lstlisting}
    
\end{multicols}
    

アスタリスクをつけることでナンバリングを外すことも可能です.
\begin{multicols}{2}
\begin{theorem*}[Name of theorem (optional)]
    Your theorem here.
\end{theorem*}

\columnbreak

\begin{lstlisting}
\begin{theorem*}[Name of theorem (optional)]
    Your theorem here.
\end{theorem*}
\end{lstlisting}
    
\end{multicols}


\newpage
\section{色}
\foreach \nycolor in \nycolors{%
\colorsample{\nycolor}
}
\newpage
\foreach \nycolor in \bootstrapcol{%
\colorsample{\nycolor}
}



\section{ハイライト}

\begin{multicols}{2}
    \newcommand{\highlighttest}[1]{\highlight[#1]{#1}}
    \foreach \nycolor in \nycolors{%
        \highlighttest{\nycolor}
    }
    \columnbreak
\begin{lstlisting}
\highlight[<color>]{<text>}
\end{lstlisting}
\end{multicols}

\begin{lstlisting}
\begin{align*}
    f(\mathbf{w}, b) =
        \highlightcap[nyRed]{\displaystyle \sum_{i=1}^k \left(y_i - \mathbf{w}^\top \mathbf{x}_i - b \right)^2}{経験誤差}
        +
        \highlightcap[nyBlue]{\displaystyle \frac{\lambda}{2} \left\|\mathbf{w} \right\|^2}{正則化項}
\end{align*}
\end{lstlisting}

\begin{align*}
    f(\mathbf{w}, b) =
        \highlightcap[nyRed]{\displaystyle \sum_{i=1}^k \left(y_i - \mathbf{w}^\top \mathbf{x}_i - b \right)^2}{経験誤差}
        +
        \highlightcap[nyBlue]{\displaystyle \frac{\lambda}{2} \left\|\mathbf{w} \right\|^2}{正則化項}
\end{align*}


\newpage

\section{枠囲み}

\begin{multicols}{2}
\begin{nyCheck}{ポイント}
    \jalipsum[3]{preamble}
\end{nyCheck}
\columnbreak
\begin{lstlisting}
\begin{nyCheck}{<title>}[<options>]
    <content>
\end{nyCheck}
\end{lstlisting}
\end{multicols}

\begin{multicols}{2}
\begin{nyAttention}{試験に出る}
    \jalipsum[3]{preamble}
\end{nyAttention}
\columnbreak
\begin{lstlisting}
\begin{nyAttention}{<title>}[<options>]
    <content>
\end{nyAttention}
\end{lstlisting}
\end{multicols}

\begin{multicols}{2}
\begin{nyHint}{ヒント}
    \jalipsum[3]{preamble}
\end{nyHint}
\columnbreak
\begin{lstlisting}
\begin{nyHint}{<title>}[<options>]
    <content>
\end{nyHint}
\end{lstlisting}
\end{multicols}

\begin{multicols}{2}
\begin{nyMemo}{メモ}
    \jalipsum[3]{preamble}
\end{nyMemo}
\columnbreak
\begin{lstlisting}
\begin{nyMemo}{<title>}[<options>]
    <content>
\end{nyMemo}
\end{lstlisting}
\end{multicols}

\begin{multicols}{2}
\begin{nyClip}{参考}
    \jalipsum[3]{preamble}
\end{nyClip}
\columnbreak
\begin{lstlisting}
\begin{nyClip}{<title>}[<options>]
    <content>
\end{nyClip}
\end{lstlisting}
\end{multicols}

\begin{multicols}{2}
\begin{nyTreble}{タイトル}
    \jalipsum[3]{preamble}
\end{nyTreble}
\columnbreak
\begin{lstlisting}
\begin{nyTreble}{<title>}[<options>]
    <content>
\end{nyTreble}
\end{lstlisting}
\end{multicols}

\begin{multicols}{2}
\begin{nyEighthNote}{タイトル}
    \jalipsum[3]{preamble}
\end{nyEighthNote}
\columnbreak
\begin{lstlisting}
\begin{nyEighthNote}{<title>}[<options>]
    <content>
\end{nyEighthNote}
\end{lstlisting}
\end{multicols}

\begin{multicols}{2}
\begin{nyGridbox}{吾輩は猫である}
    \jalipsum[1]{wagahai}
\end{nyGridbox}
\columnbreak
\begin{lstlisting}
\begin{nyGridbox}{<title>}[<options>]
    <content>
\end{nyGridbox}
\end{lstlisting}
\end{multicols}

\begin{multicols}{2}
\nyMemoFill{メモ}
\columnbreak
\begin{lstlisting}
\nyMemoFill{<title>}
\end{lstlisting}
ページ下部まで自動的に高さを調整します.
\end{multicols}

\newpage

\begin{multicols}{2}
\begin{attention}{タイトル}
    \jalipsum[3]{preamble}
\end{attention}
\columnbreak
\begin{lstlisting}
\begin{attention}{<title>}[<options>]
    <content>
\end{attention}
\end{lstlisting}
\end{multicols}

\begin{multicols}{2}
\begin{subsection1}{タイトル}
    \jalipsum[3]{preamble}
\end{subsection1}
\columnbreak
\begin{lstlisting}
\begin{subsection1}{<title>}[<options>]
    <content>
\end{subsection1}
\end{lstlisting}
\end{multicols}

\begin{multicols}{2}
\begin{subsection2}{タイトル}
    \jalipsum[3]{preamble}
\end{subsection2}
\columnbreak
\begin{lstlisting}
\begin{subsection2}{<title>}[<thickness>][<options>]
    <content>
\end{subsection2}
\end{lstlisting}
\end{multicols}

\begin{multicols}{2}
\begin{supplement1}{タイトル}
    \jalipsum[3]{preamble}
\end{supplement1}
\columnbreak
\begin{lstlisting}
\begin{supplement1}{<title>}[<thickness>][<options>]
    <content>
\end{supplement1}
\end{lstlisting}
\end{multicols}

\begin{multicols}{2}
    \begin{supplement2}{タイトル}[サブタイトル]
    \jalipsum[3]{preamble}
\end{supplement2}
\columnbreak
\begin{lstlisting}
\begin{supplement2}{<title>}[<subtitle>][<thickness>][<options>]
    <content>
\end{supplement2}
\end{lstlisting}
\end{multicols}

\newpage
\begin{multicols}{2}
\begin{mySection1}{1}{タイトル}
    \jalipsum[3]{preamble}
    \tcbline
    \jalipsum[3]{preamble}
\end{mySection1}
\columnbreak
\begin{lstlisting}
\begin{mySection1}{<num>}{<title>}
    <content>
    \tcbline
    <content>
\end{mySection1}
\end{lstlisting}
\end{multicols}

\begin{multicols}{2}
\begin{mySection2}{1}{タイトル}
    \jalipsum[3]{preamble}
    \tcblower
    \jalipsum[3]{preamble}
\end{mySection2}
\columnbreak
\begin{lstlisting}
\begin{mySection2}{<num>}{<title>}
    <content>
    \tcblower
    <content>
\end{mySection2}
\end{lstlisting}
\end{multicols}



\newpage
\begin{multicols}{2}
\begin{alert}{primary}
A simple primary alert—check it out!
\end{alert}
\begin{alert}{secondary}
A simple secondary alert—check it out!
\end{alert}
\begin{alert}{success}
A simple success alert—check it out!
\end{alert}
\begin{alert}{info}
A simple info alert—check it out!
\end{alert}
\begin{alert}{warning}
A simple warning alert—check it out!
\end{alert}
\begin{alert}{danger}
A simple danger alert—check it out!
\tcblower
You can draw a line!
\end{alert}
\columnbreak
\begin{lstlisting}
\begin{alert}{primary}
    A simple primary alert—check it out!
\end{alert}
\begin{alert}{secondary}
    A simple secondary alert—check it out!
\end{alert}
\begin{alert}{success}
    A simple success alert—check it out!
\end{alert}
\begin{alert}{info}
    A simple info alert—check it out!
\end{alert}
\begin{alert}{warning}
    A simple warning alert—check it out!
\end{alert}
\begin{alert}{danger}
    A simple danger alert—check it out!
    \tcblower
    You can draw a line!
\end{alert}
\end{lstlisting}
\end{multicols}



\begin{multicols}{2}
\begin{card}{Card}
Some quick example text to build on the card title and make up the bulk of the card's content.
\end{card}
\columnbreak
\begin{lstlisting}
\begin{card}{Card}
    Some quick example text to build on the card title and make up the bulk of the card's content.
\end{card}
\end{lstlisting}
\end{multicols}

\begin{multicols}{2}
\myTag{タグ}
\columnbreak
\begin{lstlisting}
\myTag{タグ}
\end{lstlisting}
\end{multicols}




\end{document}
